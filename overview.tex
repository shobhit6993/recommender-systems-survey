\section{Overview of Recommender Systems}
Origin of recommendation systems can be traced back to work in cognitive science [1:87], principles of forecasting [1:6] and study of marketing models [1:60]. The formal research in recommendation systems kicked off in 1990s where the backbone of recommendations was the rating framework. The problem was perceived as the problem of rating estimation for undiscovered items for a user, based on his interests, past-purchases/activities; later [38] proposed the inclusion of activities of other users with similar interests as a parameter.

[1:45], [2:86], [3:97] provided mathematical formulation for the recommendation problem. Mathematically, the recommendation system can be formalized as follows: let $\textbf{U}$ denote the user space comprising of millions of users and $\textbf{I}$ denote the item-space encompassing millions of items to be recommended. Let $r:U\times I\rightarrow R$ denote the relevance function that captures the potential relevance of an item $i\in \textbf{I}$ to a user $u\in \textbf{U}$, where $R$ represents the set of real numbers, quantifying the relevance quotient. For each user $u\in \textbf{U}$, the problem of recommendation boils down to returning a set of items $i'\in \textbf{I}$ that maximizes the relevance function, i.e.
\begin{equation}
~i'\in\max_{\forall i\in I}r(u,i)
\end{equation}
The most commonly used relevance function $r$ is the ratings provided by the user on a predetermined scale. The core problem in recommendation systems is to \textit{learn} the relevance measure for item $i$ which is either undiscovered or unrated by user $u$. Recommendation techniques can be classified into 3 main categories as follows, based on the underlying principle used to estimate the ratings:
\begin{itemize}
\item \textit{Collaborative filtering techniques} leverage the ratings of items by a set of users having similar tastes and interests as the given user $u$, and recommend items to $u$ based on these analysis. This techniques requires generation of user profiles and item profiles, and employs different similarity metrics to gauge the similarity of two users. 
\item \textit{Content-based recommendations} employ measures which are local to a user $u$. It recommends items to the user based on his own browsing/rating/purchasing patterns in the past. It envisages the recommendation problem as finding similar or related items [cite 6th paper], learning the user's tastes by analysing past activities and recommending items related to his past purchases. Here also, similarity measures are applied on item profiles to quantify similarity of two items.
\item \textit{Hybrid techniques} combines the collaborative and content-based methods to build more robust and accurate recommendation systems.
\end{itemize}
Before diving into the technicalities of the above methods, we will explore few similarity measures commonly used to quantify the \textit{closeness} of two item-item/user-user/item-user profiles.