\section{Introduction}
Recommender systems has been a hot research topic since 1990s. A lot of research has been done over the past two decades both in industry and in academia. With data growing at an enormous rate (thanks to WWW), researchers are now trying to find out new ways to utilize the available data to extract out meaningful information.  Recommender systems has been an area of high interest because of high application in e-commerce and advertisement industry products. Today almost all the big websites depend on recommender systems for suggesting ads, products, articles, movies, videos etc. to their users. Examples include book suggestions on Amazon, movie recommendation on Netflix, video suggestions on YouTube, product recommendation of Flipkart and question/answer recommendations on Quora. 

Over the past two decades, a number of different approaches have been used. Collaborative filtering is one of the most widely used approach in the making of a recommender system. It works by finding similar users and then recommending items that are liked by similar users. This approach is called \textit{user-based approach}. Another common approach in building of recommender systems is to employ filtering based on content. In content based filtering, a description of products is constructed (in the form of item matrix) based on keywords and tags. Similarly a user profile is created to indicate what a particular user likes. Then the system tries to find items similar to those that user has liked/purchased in the past. This approach is called \textit{item-based approach}. In the recent years, researchers have tried to come up with a \textit{Hybrid approach} which tries to combine collaborative filtering and content based filtering. 

However, even after a couple of decades of work in the area of recommendations, the present generation of recommendation systems require deeper investigation and research so as to make them more accurate and effective. The shortcomings of today's recommendation systems can be attributed to the enormous amount of information pervading the Internet, not only in terms of size of content, but also its variety, spectrum and diversity. Potential improvements to current methods include broadening the domain for generating user profile by analysing his activities on different social platforms, improving item-profile modelling by incorporating user feedback and contextual information (such as higher weightage to FIFA franchise items during FIFA World Cup), and building more reliable evaluation metrics. 

In this paper, we present an overview and background of recommendation systems in Section 2, followed by an overview of similarity measures which are extensively exploited in recommendation systems in Section 3. This is followed by discussion of standard techniques used in recommendation systems in Section 4. In Section 5, we explore the state-of-the-art and conclude by proposing directions for future work. 