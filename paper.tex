% THIS IS SIGPROC-SP.TEX - VERSION 3.1
% WORKS WITH V3.2SP OF ACM_PROC_ARTICLE-SP.CLS
% APRIL 2009
%
% It is an example file showing how to use the 'acm_proc_article-sp.cls' V3.2SP
% LaTeX2e document class file for Conference Proceedings submissions.
% ----------------------------------------------------------------------------------------------------------------
% This .tex file (and associated .cls V3.2SP) *DOES NOT* produce:
%       1) The Permission Statement
%       2) The Conference (location) Info information
%       3) The Copyright Line with ACM data
%       4) Page numbering
% ---------------------------------------------------------------------------------------------------------------
% It is an example which *does* use the .bib file (from which the .bbl file
% is produced).
% REMEMBER HOWEVER: After having produced the .bbl file,
% and prior to final submission,
% you need to 'insert'  your .bbl file into your source .tex file so as to provide
% ONE 'self-contained' source file.
%
% Questions regarding SIGS should be sent to
% Adrienne Griscti ---> griscti@acm.org
%
% Questions/suggestions regarding the guidelines, .tex and .cls files, etc. to
% Gerald Murray ---> murray@hq.acm.org
%
% For tracking purposes - this is V3.1SP - APRIL 2009

\documentclass{acm_proc_article-sp}
\usepackage{epstopdf}
\usepackage{amsmath}
\usepackage{bm}
\usepackage{enumitem}
\usepackage{commath}
\usepackage{xcolor}
\usepackage{graphicx}
\usepackage{natbib}
\usepackage{url}

\begin{document}

\title{Fixing of thin regions to improve 3D printability}
\numberofauthors{3} 
\author{
% 1st. author
\alignauthor
Shobhit Chaurasia\\
       \affaddr{11010179}\\
       \affaddr{Indian Institute of Technology Guwahati}\\
       \email{c.shobhit@iitg.ernet.in}
% 2nd. author
\alignauthor
Harshil Lodhi\\
       \affaddr{11010179}\\
       \affaddr{Indian Institute of Technology Guwahati}\\
       \email{harshil@iitg.ernet.in}
}

\maketitle
\begin{abstract}
The use of 3D printing has rapidly expanded in the past couple of years. It is now possible to produce 3D printed objects with exceptionally high precision. However, the 3D objects might have numerous \textit{thin} regions and weak joints, which are susceptible to breakage during the printing process itself, or during handling. We present a system that addresses this issue by automatic detection and correction of the thin regions.

The input 3D model is first voxelized. The voxel grid is scanned along different directions to identify thin regions. The thin regions, thus identified, are fixed by adding extra voxels at the violations. Further, tools from mathematical morphology are used to add extra material at joints of the 3D model. Finally, the voxelized fixed model is converted back to \texttt{STL} form using tools from \cite{abfab3d:project} project.
\end{abstract}

\section{Introduction}
Recommender systems has been a hot research topic since 1990s [c]. A lot of research has been done over the past two decades both in industry and in academia. With data growing at an enormous rate [c] (thanks to WWW), researchers are now trying to find out new ways to utilize the available data to extract out meaningful information.  Recommender systems has been an area of high interest because of high application in e-commerce and advertisement industry products. Today almost all the big websites depend on recommender systems for suggesting ads, products, articles, movies, videos etc. to their users. Examples include book suggestions on Amazon, movie recommendation on Netflix, Video recommendations on YouTube etc, product recommendation of Flipkart, question/answer recommendations on Quora. 

Over the past two decades, a lot of different approaches have been used. Collaborative filtering is one of the most widely used approach in the making of a recommender system. It works by finding similar users and then recommending items that are liked by similar users. This approach is called \textit{user-based approach}. Another common approach in building of recommender systems is content-based filtering. In content based filtering, a description of products is constructed (in the form of item matrix) based on keywords and tags. Similarly a user profile is created to indicate what a particular user likes. Then the system tries to find items similar to those that user has liked/purchased in the past. This approach is called \textit{item-based approach}. In the recent years, researchers have tried to come up with a \textit{Hybrid approach} which tries to combine collaborative filtering and content based filtering. 

However, even after a couple of decades of work in the area of recommendations, the present generation of recommendation systems require deeper investigation and research so as to make them more accurate and effective. The shortcomings of today's recommendation systems can be attributed to the enormous amount of information pervading the Internet, not only in terms of size of content, but also its variety, spectrum and diversity. Potential improvements to current methods include broadening the domain for generating user profile by analysing his activities on different social platforms, improving item-profile modelling by incorporating user feedback and contextual information (such as higher weightage to FIFA franchise items during FIFA World Cup), and building more reliable evaluation metrics. 

In this paper, we present a survey of overview and background of recommendation systems in Section- 2, followed by a discussion of standard techniques used in recommendation systems in Section - . In Section - , we explore the state-of-the-art. Finally, we conclude by proposing directions for future work. 

\bibliographystyle{plainnat}
\bibliography{paper}
\end{document}
