\section{Discussion}
\subsection{Current Extensions}
There has been plethora of work in extending the standard techniques used in recommendation systems. An interesting variation proposed in [1:22], [1:35], [1:51], [1:52] is \textit{preference-based filtering}, wherein instead of predicting absolute rating values, relative ordering of items are calculated. Another direction which is being explored is incorporating \textit{contextual} information, such as the theatre, time, and accompanying friends for movie recommendations. Others have contributed to addressing the challenges of current techniques such as the sparsity problem. Proposed solutions include rewarding users for rating products and using dynamic agents to provide automatic ratings for items.
\subsection{Directions for Future Work}
The standard techniques used for building recommendation systems can be extended in many ways to increase the accuracy, scalability and robustness of of recommendation systems. These include building models which capture user and item profiles in a more comprehensive manner. As noted in [1:2], [1:8], [1:54], [1:105], current systems fail to fully exploit user and item profiles. Most of them rely on very basic profiling models, and do not utilize advanced mining techniques to mine useful purchasing sequences [1:63], both local to a customer as well as interesting patterns over all set of users.

Another possible extension to current techniques includes broadening the domain by taking into consideration contextual information such as popular social trends, spatial and temporal information, details about potential way of usage of the item under consideration, such as recommendations for movie with girlfriend. As noted above, such techniques have already been exploited in [1:3] for movie recommendations. Extending this idea to other domains such as financial recommendations could be a non-trivial and interesting extension.

As noted in [1], introducing multi-criteria ratings is another interesting extension to bolster the effectiveness of recommendation systems. For example, instead of forcing the user to give a single rating a movie, splitting the ratings into different categories such as acting, storyline, screenplay, songs, originality of script etc. could prove to be extremely useful for more focussed recommendations.

\subsection{Conclusion}
We have presented an overview of the recommendation problem and discussed the standard techniques which have been employed to this end. We discussed 3 main categories of recommendation systems, namely \textit{collaborative, content-based} and \textit{hybrid} systems. We analysed few of their shortcomings. Further, some variations of standard techniques were highlighted which have been explored by researchers to extend the capabilities of recommender systems. Finally, we propose directions for future work which could greatly enhance the effectiveness of the present generation of recommender systems.